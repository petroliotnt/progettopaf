\documentclass[a4paper,11pt]{article}
\usepackage[T1]{fontenc}  %gestisce la codifica output
\usepackage[utf8]{inputenc} %gestisce la codifica input
\usepackage[english]{babel}  %converte il documento in altre lingue l'ultima è quella predefinita. ambiente otherlanguage per cambio lingue
%\usepackage[tight,italian]{minitoc} %sottotoc
\usepackage{microtype}  %pulizia scrittura righe. \dominitoc (prima di \toc), \minitoc, \mtcskip (chapter), \adjustmtc (chapter*)
%\usepackage[binding= 5mm]{layaureo}
\usepackage{amsmath}
\usepackage{amsfonts}
\usepackage{type1cm}
\usepackage{amsthm}
\usepackage{lmodern}
%\usepackage{cancel}
\usepackage{amssymb}
\usepackage{array}
%\usepackage{fourier}
\usepackage{backref}
\usepackage{braket}
%\usepackage{biblatex}
\usepackage{booktabs}
%\usepackage{kpfonts}
\usepackage{caption}
\usepackage{multicol}
\usepackage{changepage}
\usepackage{enumitem}
\usepackage{fancyhdr}
\usepackage{float}
\usepackage[dvips, a4paper, top = 3cm, bottom = 3cm, left = 3.5cm, right = 3.5 cm,%
heightrounded, bindingoffset=5mm]{geometry}
\usepackage{graphicx}
\usepackage{listings}
\usepackage{longtable}
\usepackage{makeidx}
%\usepackage{showlabels}
%\usepackage{quoting}
\usepackage{subfig}
\usepackage{xcolor}
\usepackage{epigraph}
\usepackage{titling}
\usepackage{mathtools}
\title{Progetto PAF}
\author{A. Baldazzi X. Turkeshi}
\begin{document}
\maketitle

\section*{Proposal}

\begin{abstract} Le osservabili quantistiche sono operatori autoaggiunti; la conoscenza della matematica che c'è dietro
arricchisce il background di un buon fisico e lo rende più preparato agli argomenti che si snocciolano nei corsi. Abbiamo deciso
di trattare questioni volte ad arricchire il percorso sia di metodi che di meccanica quantistica.

Nell'eventualità come progetto è troppo corpulento, tagliamo eventualmente i primi due punti, facciamo richiami del terzo ed esponiamo il punto iv e v.
\end{abstract}


\begin{itemize}
\item[i]  Richiami di spazi lineari topologici, spazi di Banach, di Hilbert, topologie deboli e forti;
\item[ii] Richiami di teoria della misura: motivazioni, misure importanti (p.e. di Borel e di Baire), integrale di Lebesgue-Stieltjes e teoremi importanti;
\item[iii] Operatori limitati e non limitati in spazi di Hilbert: proprietà notevoli degli operatori limitati; operatori aggiunti ed autoaggiunti;
\item[iv] Misure a valori di proiezione, teorema spettrale (di struttura degli operatori autoaggiunti)
\item[v] Applicazioni.
\end{itemize}
\end{document}
